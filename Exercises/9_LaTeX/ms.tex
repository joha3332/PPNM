\documentclass[english,a4paper,twocolumn,article,10pt]{memoir}		
\usepackage[utf8x]{inputenc}	
\usepackage{graphicx}		

\usepackage{mathtools}		


% Format of infinitesimal fx. as integration-variable
\newcommand\dx[1]{\,\text{d}#1}


\title{Besselfunctions of the first kind} 

\author{Johannes Kruse}
\date{\today}


\begin{document}

\maketitle

Bessel functions, first defined by the mathematician Daniel Bernoulli and then generalized by Friedrich Bessel, are canonical solutions y(x) of Bessel's differential equation


\begin{equation}
x^2 \frac{{\dx{}}^2 y}{{\dx{x}}^2} + x \frac{\dx{} y}{\dx{x}} + (x^2 + {\alpha}^2) y = 0
\end{equation}

for an arbitrary complex number $\alpha$, the order of the Bessel function. Although $\alpha$ and −$\alpha$ produce the same differential equation, it is conventional to define different Bessel functions for these two values in such a way that the Bessel functions are mostly smooth functions of $\alpha$.

The most important cases are when $\alpha$ is an integer or half-integer. Bessel functions for integer $\alpha$ are also known as cylinder functions or the cylindrical harmonics because they appear in the solution to Laplace's equation in cylindrical coordinates. Spherical Bessel functions with half-integer $\alpha$ are obtained when the Helmholtz equation is solved in spherical coordinates.

\begin{figure}[h]
	% GNUPLOT: LaTeX picture with Postscript
\begingroup
  \makeatletter
  \providecommand\color[2][]{%
    \GenericError{(gnuplot) \space\space\space\@spaces}{%
      Package color not loaded in conjunction with
      terminal option `colourtext'%
    }{See the gnuplot documentation for explanation.%
    }{Either use 'blacktext' in gnuplot or load the package
      color.sty in LaTeX.}%
    \renewcommand\color[2][]{}%
  }%
  \providecommand\includegraphics[2][]{%
    \GenericError{(gnuplot) \space\space\space\@spaces}{%
      Package graphicx or graphics not loaded%
    }{See the gnuplot documentation for explanation.%
    }{The gnuplot epslatex terminal needs graphicx.sty or graphics.sty.}%
    \renewcommand\includegraphics[2][]{}%
  }%
  \providecommand\rotatebox[2]{#2}%
  \@ifundefined{ifGPcolor}{%
    \newif\ifGPcolor
    \GPcolortrue
  }{}%
  \@ifundefined{ifGPblacktext}{%
    \newif\ifGPblacktext
    \GPblacktexttrue
  }{}%
  % define a \g@addto@macro without @ in the name:
  \let\gplgaddtomacro\g@addto@macro
  % define empty templates for all commands taking text:
  \gdef\gplbacktext{}%
  \gdef\gplfronttext{}%
  \makeatother
  \ifGPblacktext
    % no textcolor at all
    \def\colorrgb#1{}%
    \def\colorgray#1{}%
  \else
    % gray or color?
    \ifGPcolor
      \def\colorrgb#1{\color[rgb]{#1}}%
      \def\colorgray#1{\color[gray]{#1}}%
      \expandafter\def\csname LTw\endcsname{\color{white}}%
      \expandafter\def\csname LTb\endcsname{\color{black}}%
      \expandafter\def\csname LTa\endcsname{\color{black}}%
      \expandafter\def\csname LT0\endcsname{\color[rgb]{1,0,0}}%
      \expandafter\def\csname LT1\endcsname{\color[rgb]{0,1,0}}%
      \expandafter\def\csname LT2\endcsname{\color[rgb]{0,0,1}}%
      \expandafter\def\csname LT3\endcsname{\color[rgb]{1,0,1}}%
      \expandafter\def\csname LT4\endcsname{\color[rgb]{0,1,1}}%
      \expandafter\def\csname LT5\endcsname{\color[rgb]{1,1,0}}%
      \expandafter\def\csname LT6\endcsname{\color[rgb]{0,0,0}}%
      \expandafter\def\csname LT7\endcsname{\color[rgb]{1,0.3,0}}%
      \expandafter\def\csname LT8\endcsname{\color[rgb]{0.5,0.5,0.5}}%
    \else
      % gray
      \def\colorrgb#1{\color{black}}%
      \def\colorgray#1{\color[gray]{#1}}%
      \expandafter\def\csname LTw\endcsname{\color{white}}%
      \expandafter\def\csname LTb\endcsname{\color{black}}%
      \expandafter\def\csname LTa\endcsname{\color{black}}%
      \expandafter\def\csname LT0\endcsname{\color{black}}%
      \expandafter\def\csname LT1\endcsname{\color{black}}%
      \expandafter\def\csname LT2\endcsname{\color{black}}%
      \expandafter\def\csname LT3\endcsname{\color{black}}%
      \expandafter\def\csname LT4\endcsname{\color{black}}%
      \expandafter\def\csname LT5\endcsname{\color{black}}%
      \expandafter\def\csname LT6\endcsname{\color{black}}%
      \expandafter\def\csname LT7\endcsname{\color{black}}%
      \expandafter\def\csname LT8\endcsname{\color{black}}%
    \fi
  \fi
    \setlength{\unitlength}{0.0500bp}%
    \ifx\gptboxheight\undefined%
      \newlength{\gptboxheight}%
      \newlength{\gptboxwidth}%
      \newsavebox{\gptboxtext}%
    \fi%
    \setlength{\fboxrule}{0.5pt}%
    \setlength{\fboxsep}{1pt}%
\begin{picture}(3940.00,2800.00)%
    \gplgaddtomacro\gplbacktext{%
      \csname LTb\endcsname%%
      \put(202,667){\makebox(0,0)[r]{\strut{}$-0,4$}}%
      \csname LTb\endcsname%%
      \put(202,881){\makebox(0,0)[r]{\strut{}$-0,2$}}%
      \csname LTb\endcsname%%
      \put(202,1096){\makebox(0,0)[r]{\strut{}$0$}}%
      \csname LTb\endcsname%%
      \put(202,1310){\makebox(0,0)[r]{\strut{}$0,2$}}%
      \csname LTb\endcsname%%
      \put(202,1524){\makebox(0,0)[r]{\strut{}$0,4$}}%
      \csname LTb\endcsname%%
      \put(202,1738){\makebox(0,0)[r]{\strut{}$0,6$}}%
      \csname LTb\endcsname%%
      \put(202,1953){\makebox(0,0)[r]{\strut{}$0,8$}}%
      \csname LTb\endcsname%%
      \put(202,2167){\makebox(0,0)[r]{\strut{}$1$}}%
      \csname LTb\endcsname%%
      \put(207,385){\makebox(0,0){\strut{}$0$}}%
      \csname LTb\endcsname%%
      \put(620,385){\makebox(0,0){\strut{}$1$}}%
      \csname LTb\endcsname%%
      \put(1033,385){\makebox(0,0){\strut{}$2$}}%
      \csname LTb\endcsname%%
      \put(1446,385){\makebox(0,0){\strut{}$3$}}%
      \csname LTb\endcsname%%
      \put(1859,385){\makebox(0,0){\strut{}$4$}}%
      \csname LTb\endcsname%%
      \put(2272,385){\makebox(0,0){\strut{}$5$}}%
      \csname LTb\endcsname%%
      \put(2685,385){\makebox(0,0){\strut{}$6$}}%
      \csname LTb\endcsname%%
      \put(3098,385){\makebox(0,0){\strut{}$7$}}%
      \csname LTb\endcsname%%
      \put(3511,385){\makebox(0,0){\strut{}$8$}}%
      \csname LTb\endcsname%%
      \put(3924,385){\makebox(0,0){\strut{}$9$}}%
    }%
    \gplgaddtomacro\gplfronttext{%
      \csname LTb\endcsname%%
      \put(91,1417){\rotatebox{-270}{\makebox(0,0){\strut{}y}}}%
      \csname LTb\endcsname%%
      \put(2065,123){\makebox(0,0){\strut{}x}}%
      \csname LTb\endcsname%%
      \put(2065,2537){\makebox(0,0){\strut{}Besselfunctions}}%
      \csname LTb\endcsname%%
      \put(3819,2116){\makebox(0,0)[r]{\strut{}Bessel 𝜈=0}}%
      \csname LTb\endcsname%%
      \put(3819,1941){\makebox(0,0)[r]{\strut{}Bessel 𝜈=1}}%
      \csname LTb\endcsname%%
      \put(3819,1766){\makebox(0,0)[r]{\strut{}Bessel 𝜈=2}}%
      \csname LTb\endcsname%%
      \put(3819,1591){\makebox(0,0)[r]{\strut{}data 𝜈=0}}%
      \csname LTb\endcsname%%
      \put(3819,1416){\makebox(0,0)[r]{\strut{}data 𝜈=1}}%
      \csname LTb\endcsname%%
      \put(3819,1241){\makebox(0,0)[r]{\strut{}data 𝜈=2}}%
    }%
    \gplbacktext
    \put(0,0){\includegraphics{Bessel}}%
    \gplfronttext
  \end{picture}%
\endgroup

	\caption{Illustration of the exponential function.}
\end{figure}



\end{document}